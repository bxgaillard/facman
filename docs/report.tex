\NeedsTeXFormat{LaTeX2e}

\documentclass[a4paper,12pt,oneside]{article}
\usepackage[latin1]{inputenc}
\usepackage[french]{babel}
\usepackage[T1]{fontenc}
\usepackage{ae,aecompl,aeguill}
\usepackage{geometry}

\begin{document}

\setlength{\parskip}{.5 \baselineskip}

\begin{center}
\sc\LARGE
Projet Java\\
Jeu : Facman\\
\rule[5pt]{.5\textwidth}{2pt}\\
Rapport
\end{center}
\vspace*{1cm}

\section*{Pr�sentation}

Le jeu que nous avons choisi de r�aliser est un Pacman. Ce dernier reprend les
id�es originales du jeu : un personnage se d�place dans un labyrinthe parsem�
de pastilles, il doit toutes les manger afin de passer au niveau suivant. Afin
de corser la t�che, des fant�mes hantent ce labyrinthe. Les collisions avec
ces derniers sont fatales, il faut soit les �viter soit les manger gr�ce �
l'utilisation de super pastilles.

\section*{D�tail des classes}

Ce projet est compos� de 11 classes :
\begin{itemize}
\item classe \emph{Facman} : chef d'orchestre des autres classes, elle
  contient le ``main'' et fait appel aux diff�rents �l�ments du jeu ;
\item classe \emph{Character} : g�re les caract�ristiques communes aux
  personnages, Pacman et Ghost ;
\item classe \emph{Pacman} : d�rive de Character, g�re les possibilit�s de
  d�placement et l'affichage du personnage principal du jeu ;
\item classe \emph{Ghost} : fille de Character, contr�le les monstres et leur
  affichage ;
\item classe \emph{Position} : stocke et actualise les coordonn�es des
  personnages ;
\item classe \emph{Board} : construit le labyrinthe � partir d'un fichier
  texte repr�sentant le niveau ;
\item classe \emph{ImageLoader} : d�coupe les images sous forme d'ic�nes et
  les charge en m�moire. Chaque personnage est � l'origine repr�sent� sous
  toutes ses faces dans une seule image ;
\item classe \emph{Game} : contr�le la mise en pause, l'affichage du menu, du
  score, du nombres de vies, et les niveaux de difficult� ;
\item classe \emph{HomeScreen} : g�n�re l'�cran d'accueil ;
\item classe \emph{AppletImageLoader} : permet le chargement des images dans
  le cadre d'un applet ;
\item classe \emph{ApplicationImageLoader} : permet le chargement des images
  dans le cadre d'une application autonome.
\end{itemize}~

Il est aussi notable que Facman peut-�tre utilis� au choix : soit en tant
qu'application, soit comme un applet � travers une page HTML.

Pour commencer le jeu, ex�cuter dans un terminal soit
\texttt{java -jar facman.jar} (d�marrage en mode application) soit
\texttt{appletviewer facman.html} (mode applet).

Have fun!

\begin{flushright}
Benjamin \textsc{Gaillard}\\
Lionel \textsc{Imbs}
\end{flushright}

\end{document}
